\chapter{Methodology}
The study was conducted using a progressive web app (pwa) that has users walk through a sequence of steps in which the two game modalities are being used and tested by the two groups over a periond of XXXXX weeks. There are 7 steps in the sequence and together these introduced the participants to the study and then carried the users through the testing. The two groups were randomly assigned one of the two game modalities, in what will be refered to as the MAIN part of the study. MAIN was the time period of the study where the bulk of the data is collected. All game data is being collected for each game played. MAIN was the segment of the study where the randomization occured whereby the groups were assigned modality A and B respectively. Each step of the study is described in explicit detail in section XXXXXX. All 12 intervals of the chromatic scale were tested in the game.


%\section{Initial Eval}
%Reiterate here that most apps work the same. I have been using apps for years.

\section{App interface}
\subsection{Screens}
There are basically 3 screens (dashboard/introduction/game)in the app. The app has a basic login screen and a dashboard where users can choose to either play the game or view the introduction to the study. The introductory display is always available in the dashboard. If user decides to play the game, then the user is taken to the point where she left off last time. Again each step is described in section XXXXX. Each game is 3 minutes long.

\subsection{Sound stimulus}
During the study a standard piano sound was used in the game. The sound stimuly or piano note has a length of, or rather, a decay of 500ms. It is quite a short note but the pitch is clearly audible. Interval notes are randomly selected from within the range C2 to C6.

\subsection{Game interaction}
Players use a 3 by 4 grid to input intervals (see appendix for images) that corresponds to the 12 chromatic intervals starting from the lower left with minor second interval leading up to a perfect octave in the upper right corner. Users input their answers by either tapping the corresponding interval square in the grid or by dragging their finger over the interval. If a user touched the correct interval, then it glows in a green color and if the user touches an incorrect interval, then it glows red. This means that if a user dragged her finger over the grid then many colors will glow on and off depending on which interval square the user is touching. This design was implemented in order to allow for unlimited speed in the game. This matters primarily in the case of connected intervals, described in the next section, because each interval builds on the previously played note and therefore one has the ability to play the game and discriminate intervals at a much higher pace than with the traditional mode. Having the ability to drag ones finger over the grid allows the user
to input intervals faster than if the user would have to lift her finger everytime she wants to input a new interval.


% CREATE A GRAPHIC THAT SHOWS ALL SCREENS
% \begin{figure}[H]
% \centering
% \includegraphics[width=0.26\textwidth]{chapters/Fig/soxhletfiltration.png}
% \textbf{\emph{\caption{\label{fig:soxhletfiltration}Soxhelt filtration set up \cite{72}.}}}
% \end{figure}


\section{Game Modalities}

\subsection{Traditional (separated intervals)}
%  Traditionally most ear training apps that aim to improve categorical discrimination of intervals (or other aspects) usually are based on three steps that are being repeated over and over again every time the
%  user answers correctly.
Traditional separated intervals means that initially, the sound stimulus is played back to the user; Secondly, the user tries to discriminate the category of the stimulus; Thirdly, if the user answers correctly, then there is a short pause before the procedure repeats itself. In short this is [question; answer; pause] or [stimulus ; userinput ; wait].

\subsection{Connected or chained intervals}
Connected intervals is a new method for practicing interval discrimination which also is mentionned in the introductory section of this paper. In this mode only one note is played back to the user at a time and it is up to the user to now discriminate the interval between this note and the previous note. This means that when the game starts two notes are required to be played back to the user so that the user has two notes to compare and then going forward only single notes are required. When the user presses "start" then two notes are played back (note A and B), and now if the user inputs this interval AB successfully, going forward only a note C is required. Now the goal for the user becomes to input the interval BC, and so on and so forth. Notice that this requires the user to hold the note B in short-term memory so that she can reproduce the interval BC in her head and move forward in the game by inputting the interval CD, and then DE, etc..

\subsection{Game modality implementation.}
In reality, the traditional mode was implemented without a pause. This actually worked but now the difference between separated intervals and connected became one piano note. In the separated mode
it is required to to hear two notes for each queston, and with connected intervals you only need to hear one note per question. Each piano note being approximately 500ms makes the ceiling by which users can answer > 500ms and 0ms for connected intervals. In the end this meant that the study was testing a group of humans ability to discriminate categorical unrelated (atonal) musical intervals where one group is answering based on a 500ms ceiling + response time, and the other group answering for the same amount of data but only having to consider the response time and having to use their short-term memory.

(some of the above information should be moved to the introductory chapter.)



% CREATE A GRAPHIC THAT EXPLAINS EACH MODE
% \begin{figure}[H]
% \centering
% \includegraphics[width=0.26\textwidth]{chapters/Fig/soxhletfiltration.png}
% \textbf{\emph{\caption{\label{fig:soxhletfiltration}Soxhelt filtration set up \cite{72}.}}}
% \end{figure}


\section{Study Protocol}
In the following section each step of the study is described in detail.

\subsection{step 1: INTRO}
Here participants were introduced to the subject. User proceeds to next step by pressing the `next` button.

\subsection{step 2: FORM}
At this step basic information about the participants were collected with the goal of facilitating the analysis later on. The questons used were the following: 1. Do you play an instrument or sing?; 2. Have you ever studied music theory?; 3. Have you used any tools for ear training before?; 4. What is your age group?
Again, user proceeds to next step by pressing the `next` button.

\subsection{step 3: PRE}
Now users enter the actual game. The first time users play the game each and every participant will play the traditional game mode with a pause. The data collected at this stage and in stage 5 (POST) will act as diagnostic reference points.

\subsection{step 4: MAIN}
This is where the majority of the time was spent by all participants. Here the game modalities were randomly distributed amongst participants.

\subsection{step 5: POST}
At this very last stage of the game all users were assigned the traditional game modality again.

\subsection{step 6: FINISHED}
Gratitude was expressed towards all participants who made it this far. Now users are also informed that the app is made available to them for free should they want to continue using the app.
%Participants were thanked for their contribution to the sciences.

\subsection{step 7: DONE}
The app was made available, and is, for free.

\section{Participants}
A dedicated digital poster was used in order to catch people's eyes and recruite participants for the study. This poster was sent of to a variety of people including people in music schools, music business, and musically interested individuals in general. Initially the poster was intended to be put in physical location but due to COVID19 the poster was sent out digitally instead.
% Some people were recruite online and some people saw the poster in an actual physical location at a school as an example.
A facebook support page was created as well in order to support participants that did not fully understand how to play the game
or follow the in game instructions. https://www.facebook.com/PitchMachineApp


\section{Data Collection and Analysis}
The time delta for each answer is collected along with the correct interval in each session as an array. All elements of the array summed is equal to the total time of a game played. Unfinished games were discarded. Game data is visualized in the following sections with python. Standard statistical analysis is performed on data.

% \section{method critique}}
% 1. allowing users to drag or slide randomly might induce garbage data but my hypothesis is that

