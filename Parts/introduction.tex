\chapter{Introduction}

\section{Background}
One of the most important aspects of training for musicians are their ears. A musicians job is to deal with the auditory perception, and in vernacular english improving one's auditory perception is commonly referred to as ear training. However, ear training can be broken down into many different categories of perceptual training, some of which are very important to continually practice for musicians and others who are less important.
One important aspect of ear training to musicians is relative interval identification. When musicians practice relative interval identification, the goal is to learn how to recall and identify different musical intervals. In western tonal harmony there are 12 notes and hence there are also only 12 different musical intervals. On a keyboard each interval can be produced, eg. by playing a low C note on a piano and then play each note of the octave simultaneously with it until you reach the C one octave above. If you continue above C one octave above, you will perceive the same intervals all over again untill you reach the next C octave. This phenomenon is called categorical perception. Categorical perception (CP) is a mechanism whereby non-identical stimuli that have the same underlying meaning become invariantly represented in the brain \cite{klein2011role}.

Relative interval identification is one aspect of auditory perception that is a categorical perception and the goal of the present study is to evaluate two different interactive methods for training relative pitch identification on the set of western tonal musical intervals, i.e. the notes of the piano.
One of the methods tested is based on how relative interval training is implemented in most cases on many popular mobile applications in the iOS app store. The second method is a new method designed under the hypothesis of being a more musical procedure for training relative intervals than the conventional method, and even more important is the fact that this new method gives the user ability to discriminate more and more intervals per unit time as their ability improves in contrast to the traditional modality where there is a limit to how fast users can play the game and improve themselves.

Many interactive tools for improving relative interval identification are based on the same interactive protocol.

- List of tried and tested applications during the research phase. goodEar Pro; Relative Pitch Interval Ear Training, Interval Ear Training, Piano Ear Training, Ear Trainer, Ear Beater

Initially, two notes are played back to the user that together form a musical interval. Then it is upon the user of the application to input to the application which interval the user identifies it as. If the interval is identifies correctly, there is an enforced pause before the next set of two notes forming a new interval are played back. This procedure can be summarized as a question, an answer, and then the enforced pause (QAP).
The length of the pause varies between applications but is usually around 1 second; in some applications its shorter and others a few tenths of milliseconds longer.
This pause is interesting because the length of the pause lies within a temporal range that is close to the pace at which musical intervals can be played in a song or melody. This suggests that removing the pause and playing back the next question instant/aneously upon answering could make training more similar to a real musical context and not just viewed as a musical training tool.
Within all the applications tested each interval played back is also separate from the previously played interval. Every time the user answers correctly two new notes are played back and each consecutive pair of notes represent one single interval and there is no relation from one question to the next. However, this has just happened to become just a convention but there is nothing that restricts one from making each question dependent on the previous one and by doing so making the exercise protocol more similar to how real melodies are played. This is how the second method works, from here on referred to as connected intervals (as opposed to traditional separated intervals).
Connected intervals is a protocol for categorical learning and identification of musical intervals. Each training session is initialized with two notes played back representing one interval. After the identification of this initial interval the next note is played back instantaneously upon correct user input of the first interval. Identification of the next interval is now done by comparing each consecutive note to he previous one by holding the previous note in short-term memory. Hence, only single notes are played back at each iteration of the game session after the initial two-note interval. This allows for the user to dictate the pace at which the training is done dynamically and it allows for training at very high speed should the user be very good at identifying intervals. Symbolically the protocol can be presented as follows: playback notes AB, input AB —> playback note C, input BC —> playback note D, input CD, and so on.

\section{Theory}

todo... ???
